\documentclass{beamer}
entation> {

% The Beamer class comes with a number of default slide themes
% which change the colors and layouts of slides. Below this is a list
% of all the themes, uncomment each in turn to see what they look like.

%\usetheme{default}
%\usetheme{AnnArbor}
%\usetheme{Antibes}
%\usetheme{Bergen}
%\usetheme{Berkeley}
%\usetheme{Berlin}
%\usetheme{Boadilla}
%\usetheme{CambridgeUS}
%\usetheme{Copenhagen}
%\usetheme{Darmstadt}
%\usetheme{Dresden}
%\usetheme{Frankfurt}
%\usetheme{Goettingen}
%\usetheme{Hannover}
%\usetheme{Ilmenau}
%\usetheme{JuanLesPins}
%\usetheme{Luebeck}
\usetheme{Madrid}
%\usetheme{Malmoe}
%\usetheme{Marburg}
%\usetheme{Montpellier}
%\usetheme{PaloAlto}
%\usetheme{Pittsburgh}
%\usetheme{Rochester}
%\usetheme{Singapore}
%\usetheme{Szeged}
%\usetheme{Warsaw}

% As well as themes, the Beamer class has a number of color themes
% for any slide theme. Uncomment each of these in turn to see how it
% changes the colors of your current slide theme.

%\usecolortheme{albatross}
%\usecolortheme{beaver}
%\usecolortheme{beetle}
%\usecolortheme{crane}
%\usecolortheme{dolphin}
%\usecolortheme{dove}
%\usecolortheme{fly}
%\usecolortheme{lily}
%\usecolortheme{orchid}
%\usecolortheme{rose}
%\usecolortheme{seagull}
%\usecolortheme{seahorse}
%\usecolortheme{whale}
%\usecolortheme{wolverine}

%\setbeamertemplate{footline} % To remove the footer line in all slides uncomment this line
%\setbeamertemplate{footline}[page number] % To replace the footer line in all slides with a simple slide count uncomment this line

%\setbeamertemplate{navigation symbols}{} % To remove the navigation symbols from the bottom of all slides uncomment this line
}

\usepackage{graphicx} % Allows including images
\usepackage{booktabs} % Allows the use of \toprule, \midrule and \bottomrule in tables

\title{the recent result in 1116}
\author{Guo Jun-Guang \& Li Hai-Jun}
\date{\today}

\begin{frame}
\frame{title}{Overview}
\tableofcontents
\end{frame}
%----------------------------------------------sources------------------------------
%1806.03866
%NGC 253 13 data points

%1810.12676  1708.03126  1610.08894
%HESS J1825-137
%there is Milky Way sources.
%3.9Kpc

%1708.00658
%FSRQ PKS 0736+017
%z=0.189
%no useable data points

%1807.01302
%PSR B0833-45(P2)
%~300pc  0.2kpc
%Appendix B: systematic erros on spectral parameters

%1802.00216
%gamma-ray origin


%10.22323/1.301.0730
%Binary system LMC P3
%not usefor

%10.1063/1.4772304
%Discovery of VHE γ-ray emission from the very distant BL Lac KUV 00311-1938 by H.E.S.S.
%no datas

%1708.00882
%1708.00775
%3C 279
%data points is little.
%Mrk501
%error bar is large.
%higher energy 
%z = 0.034

%PKS 0736+017
%no data

%1608.06994
%Swift J1834.9-0846
%Magnetar Wind Nebulae

%1509.03425
%Sgr A*
%MW 

%1509.03090
%PSR B1259-63
%~2.7kpc
%1509.02902
%2kpc
%data are good.
%1509.02800
%no data
%1303.1628
%discover PKS 0447-439
%ssc model

%need to see
%H.E.S.S Galactic plane survey

%-----------------------------Mrk501----------------------
%IGMF about Mrk501 has been discussed

%cluster

\section{PG 1553+113}
\begin{frame}
\frametitle{fran}

\begin{frame}
\frametitle{ciber}
\begin{block}{Block 1}
%To illustrate our method, we will make the assumption that this source is located in a galaxy cluster when considering the ICM scenarios. }
%\section{Centaurus A}
%{
%distance: Cen 3-5Mpc
%		  Cen A 4.2+-0.3 Mpc
%the nearest active galaxy.
%the spectrum in 1807.07375 is not suit for fitting.
%}
